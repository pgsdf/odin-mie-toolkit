
\documentclass[11pt]{article}
\usepackage{amsmath, amssymb, geometry, hyperref}
\geometry{margin=1in}
\title{Odin-Mie Technical Brief}
\author{Vester Thacker}
\date{\today}

\begin{document}
\maketitle

\section{Overview}
Odin-Mie implements physically based electromagnetic scattering for homogeneous spheres using Mie theory,
together with water dielectric models (MW2004, Debye) and TEOS-10 seawater conductivity. It integrates
over drop size distributions (DSD) to produce radar-observable attenuation $\gamma$ and fits power-law
relations $\gamma \approx k R^\alpha$.

\section{Mie Theory}
Define size parameter $x = 2\pi r/\lambda$ and complex refractive index $m=\sqrt{\varepsilon}$.
The efficiency factors are
\begin{align}
Q_{\mathrm{ext}} &= \frac{2}{x^2}\sum_{n=1}^{\infty} (2n+1)\Re(a_n + b_n), \\
Q_{\mathrm{sca}} &= \frac{2}{x^2}\sum_{n=1}^{\infty} (2n+1)(|a_n|^2 + |b_n|^2), \\
Q_{\mathrm{back}} &= \frac{1}{x^2}\left|\sum_{n=1}^{\infty} (2n+1)(-1)^n (a_n - b_n)\right|^2.
\end{align}
We use Riccati--Bessel recurrences with truncation $n_{\max}\approx x + 4x^{1/3} + 2$.

\section{Dielectric Models}
We write
\begin{equation}
\varepsilon(f,T,S,p) = \varepsilon_{\mathrm{pw}}(f,T) - j \frac{\sigma(S,T,p)}{\varepsilon_0 \omega}, \qquad \omega=2\pi f.
\end{equation}
\subsection{MW2004}
MW2004 supplies $\varepsilon_{\mathrm{pw}}(f,T)$ for pure water using a double-Debye form with temperature-dependent parameters.

\subsection{TEOS-10 Conductivity (PSS-78 + SAL78)}
Conductivity ratio
\begin{equation}
R = \frac{C(S,T,p)}{C(35,15^\circ\mathrm{C},0)}.
\end{equation}
At zero pressure, Practical Salinity $S_P$ is obtained from $R_t$ via PSS-78, with temperature polynomial $r_t(T_{68})$.
For $p>0$, SAL78 uses
\begin{equation}
A(X_T), \quad B(X_T), \quad C(X_P)
\end{equation}
with $X_T = T_{68}-15$, $X_P=p$ (dbar) and a quadratic solve to recover $R$.

\section{Microphysics and Attenuation}
With a drop size distribution $N(D)$ and extinction cross section $\sigma_{\mathrm{ext}}(D)$,
\begin{equation}
\kappa = \int N(D)\,\sigma_{\mathrm{ext}}(D)\,dD, \qquad \gamma~[\mathrm{dB/km}] = 4.343\times10^3 \kappa.
\end{equation}
We provide Marshall--Palmer and Gamma DSDs with Simpson quadrature over drop radius.

\section{Fitting $k,\alpha$}
Given samples $\{(R_i,\gamma_i)\}$,
\begin{equation}
\ln \gamma_i = \ln k + \alpha \ln R_i,
\end{equation}
solved by linear least squares in $(\ln R,\ln \gamma)$.

\section{Validation}
We include unit tests for the TEOS anchor $C(35,15^\circ\mathrm{C},0)=42.9140$~mS/cm and pressure/temperature monotonicity,
plus validation tables and sweeps.

\section{References}
\begin{itemize}
\item Meissner, T., \& Wentz, F. J. (2004). The complex dielectric constant of pure and sea water from microwave satellite observations. \emph{IEEE TGRS}, 42(9), 1836--1849.
\item UNESCO (1983). Algorithms for computation of fundamental properties of seawater. \emph{Tech. Papers in Marine Science 44}. (SAL78 listing.)
\item TEOS-10 Manual and GSW Toolbox (v3.06).
\item Marshall, J. S., \& Palmer, W. McK. (1948). The distribution of raindrops with size. \emph{QJRMS}.
\end{itemize}

\end{document}
